\chapter{Theory}
\section{Give a description of the theory involved for this particular lab.}
Objects that fall freely near Earth will accelerate at approximately $9.81\frac{m}{s^2}$
if they:
\begin{itemize}
  \item are near the surface of the Earth
  \item are at a geographic location where the Earth's radius is approximately its average value
  \item When air friction is ignore, the acceleration of an object falling is independent of the mass,
    volume, and shape
  \item For short distances and small objects it is not unreasonable to neglect air resistance, but for
  \item For long distances and large objects, air resistance is significant
Acceleration of gravity $g$ can be calculated using kinematics
  \item $\Delta y$ is the displacement
  \item $t$ is the time
  \item $v_0$ is the initial velocity
  \item $v_f$ is the final veolcity
  \item $a = g_{exp}$ $a$ is the experimental acceleration due to gravity
\end{itemize}

