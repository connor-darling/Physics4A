\chapter{Procedure}

\section{Procedure Equipment}

The necessary equipment for this lab is as follows:
\begin{itemize}
  \item 2 meter aluminum rod
  \item Table clamp
  \item Swivel clamp or $90^\circ$ Offset Clamp
  \item Meter Stick
  \item $\frac{1}{2}"$ and $\frac{3}{8}"$ steel ball
  \item Freefall Adapter - PASCO \#ME
  \item PASCO 750 Interface box
  \item PASCO Interface box power cord and SCSI Interface Card
  \item Laptop Computer w/ Data Studio
\end{itemize}

\section{A. Apparatus Setup}

  Setup the apparatus as shown in the picture below. Make sure that the clamps are tight,
but not over tight. The clamp holding the apparatus should be tight enough to not allow
the apparatus to move around. Makre sure that the small thumb screw on the ball
clamp is on top.

\newpage 

\begin{figure}[htbp]
  \centerline{\includegraphics[scale=0.3]{resources/photo1.jpg}}
\end{figure}

  The apparatus consists of a small clamp that holds the metal ball and a steel rod;
it is designed to complete a circuit when the ball is installed. There is a plunger
that goes through the thin steel plate that has a retaining ring on both ends.
This plunger places pressure on the steel plate and holds the ball in place.

  Place the ball in the clamp, positioning it between the hole and the brass screw, pull on
plunger so that the retaining ring is exerting pressure on the steel plate. When there is enough
pressure on the plate to retain the ball, tighten the thumbscrew. The ball and apparatus are now
ready to be used.

  Position the floor plate directly beneath the ball clamp. Do a few test drops to make sure
that the ball solidly strikes the plate. \\
\textbf{Possible Pitfalls:}

  Due to the age of the apparatus, the circuit may be easily broken. It is possible, however, to
maneuver the ball so that the circuit remains closed.
  The thumbscrew must be tightly set; otherwise it will not hold the plunger. This means that
the slightest turn on the thumbscrew will release the plunger.

\section{B. Computer Setup}

  Provide power to the PASCO 750 Interface box, connect the box to the computer with the
SCSI cord and turn on the box, making sure the word “Honda” on the cord is facing upward when
it gets plugged into the computer. \textbf{Only after you have turned on the PASCO 750 Interface
box 750 and plugged it into the computer, can you turn on the computer. (If you do not
perform these tasks in the above listed order, your computer will not recognize the PASCO
750 Interface box.)} When the computer is first turned on, it may have a Found New Hardware
window open. If so, just ignore the window and proceed with the steps below.

  Double-Click on the \emph{Data Studio} icon on the desktop. The following screen should appear:


\begin{figure}[ht]
  \centerline{\includegraphics[scale=0.45]{resources/photo2.jpg}}
\end{figure}

\newpage Click \emph{Create Experiment}, if it cannot find the Interface box right away, click \emph{Scan}.
If it doesn’t find it after you have clicked \emph{Scan}, make sure that it is connected properly.
If it is connected properly and everything seems 
to be in order, restart the computer and the Interface box should load proper.\\
Once the Interface box is found, the following screen will appear:

\begin{figure}[ht]
  \centerline{\includegraphics[scale=0.4]{resources/photo3.jpg}}
\end{figure}

This is your main window. This window contains a picture of the interface box you currently have attached, 
access to a list of sensors that can be attached to the Interface box, a toolbar to \emph{adjust} sensor options and other items.

Under \emph{Experiment Set-Up}, click \emph{Add Sensor or Instrument} and a window will open. 
Then click on the small black arrow near the top of the window to open a drop-down 
menu and select \emph{Scientific Workshop Digital Sensors}. Finally select \emph{Free-Fall Adapter} and click \emph{Ok}. See picture below.

In the \emph{Experiment Set-Up} window, on the bottom left corner of the window, 
click the \emph{Measurements} tab and de-select \emph{Acceleration, Ch 1 box}. As seen below.

\begin{figure}[t!]
  \centerline{\includegraphics[scale=0.45]{resources/photo4.jpg}}
\end{figure}

  \newpage Plug the \emph{Free Fall Adapter} into the channel indicated (here it is in Channel 1).
Click the \emph{Sampling Options} button (located on the top of the \emph{Experiment Setup} window). Make the selections as below:


\begin{figure}[h!]
  \centerline{\includegraphics[scale=0.475]{resources/photo5.jpg}}
\end{figure}

  When you click on \emph{“Keep data values only when commanded,”} you will have 
to uncheck \emph{“Enter a keyboard value when data is kept.”} Click \emph{Ok}. The computer 
is now ready to collect data.

\newpage

\section{C. Data Collection Setup}

On the left-hand side of the Data Studio program, there are two windows: \emph{Data}
and \emph{Displays}. 

\emph{Data} shows you what you are going to collect. Here, you can see that the only 
data being collected is the \emph{Time of Fall}, in seconds.

\emph{Displays} gives you several choices and formats in which to display your captured 
data. In this lab, we are only going to be concerned with the \emph{Table} display.
We will explore other displays in subsequent labs.

Double-Click \emph{Table} in Displays and select \emph{Time of Fall, Ch 1 (s)}. Click \emph{OK}. 
The following table will appear on your screen:


\begin{figure}[ht]
  \centerline{\includegraphics[scale=0.4]{resources/photo6.jpg}}
\end{figure}

  Make sure that the \emph{Show Time} icon, on the top of your table, is \textbf{off}. 
You are now ready to begin the experiment.

\section{D. Data Collection}

  The goal of this experiment is to capture the time it takes the ball to fall a certain distance.
We have set up the computer and apparatus to assist us in capturing the data required.

  You will be dropping each ball over a minimum of 5 distances 
(0.2 m, 0.4 m, 0.6 m,0.8 m, 1.0 m, etc.). Drop the ball 5 times per height 
to get an average time.

  Load the apparatus with the large ball, and set the apparatus at the correct 
height. Click the Start button; this will start the timer. Leave it on 
for the remainder of the experiment. When you drop the ball, the data will 
appear on the table. If it appears to be good data, click the \emph{Keep} button

\begin{figure}[ht]
  \centerline{\includegraphics[scale=0.6]{resources/photo7.jpg}}
\end{figure}

  \newpage Repeat 5 times for the given height and press Stop. You can rename 
the run to help keep track of where you are in the process (ie. 0.2 m, $\frac{1}{2}"$ 
Ball) by clicking on the \emph{Run} \# Label in the \emph{Data} window.

\begin{figure}[ht]
  \centerline{\includegraphics[scale=0.45]{resources/photo8.jpg}}
  \end{figure}

  This is your first set of times. Repeat the experiment at the same height with the smaller ball.\\\\
  Set the apparatus at a new height and repeat.\\\\
  Note: Every time you start and stop the timer, you create new runs.\\\\
  It is possible to delete a run if necessary, Click Experiment then Delete Last
  Data Run. Caution\: In Data Studio you can only delete the last data run or ALL 
  data runs, however, once all of the runs have been completed, you will move the 
  data from each table into Excel for analysis. In \emph{Excel} you can delete any bad data 
  runs that you were unable to delete in Data Studio.\\


  You should have at least 5 tables of times per ball, each properly labeled 
  and ready for transfer into \emph{Excel}. If the data tables are not all open, then 
  go to the left-hand side of the Data Studio program, where the \emph{Data and 
  Displays} windows are. Click (and hold) on your data run, then drag it to the 
  \emph{Table} icon under \emph{Displays} and release the mouse button. A table of that data 
  run should appear, showing the time of fall.

  \section{Data Collection Cont.}

  You should have at least 5 tables of times per ball, each properly labeled and
ready for transfer into Excel. If the data tables are not all open, then go to the
left-hand side of the Data Studio program, where the Data and Displays
windows are. Click (and hold) on your data run, then drag it to the Table icon
under Displays and release the mouse button. A table of that data run should
appear, showing the time of fall.\\\\
The table below should give you an idea of what you are looking for, however
using the method described here, you will have each data table in a separate
window.\\\\

\begin{figure}[bh!]
  \centerline{\includegraphics[scale=0.45]{resources/photo9.png}}
\end{figure}

\section{Procedural Analysis and Results}

In order to do the analysis of the data, you are going to transfer this data into \emph{Excel}. In
order to properly transfer it, move the mouse over to the cell that says \emph{Time Of Fall}, your cursor
will change into a down arrow. \textbf{Click} once to select the data in the table, \textbf{click} \emph{Edit} in the \emph{menubar},
then click \emph{Copy}; OR, type \emph{Ctrl-C} to copy. Once copied, open \emph{Excel} and start a new workbook.
Click cell \emph{A1}, go to \emph{Edit} on the \emph{menubar}, click \emph{Paste}; OR, type \emph{Ctrl-V} to paste. Two columns and a
header will appear in \emph{Excel} as shown below

\begin{figure}[h!]
  \centerline{\includegraphics[scale=0.45]{resources/photo10.png}}
\end{figure}

\textbf{Copy} or move the header (Time Of Fall, Ch 1, 0.2m, $\frac{1}{2}"$ Ball) one cell to the left and then
\textbf{delete} the first column. The first column is basically meaningless, it shows you when you took the
measurement based on the elapsed time on the timer.
  
  Repeat for each table of data you collected. By the time you are done with the copying,
pasting and deleting, your \emph{Excel} table should look like this:

\begin{figure}[h!]
  \centerline{\includegraphics[scale=0.45]{resources/photo11.png}}
\end{figure}

Using \emph{Excel}, you will now find the average time for each run. To find the average, type in the
following command into the cell directly below the 1st
run of data. =AVERAGE(A3:A7), where
A3:A7 is the range of cells over which you are calculating the average. After you have found the
average times for each run, you are now ready to begin the calculation for g. In a cell away from
the runs, begin a new table, with headers “\emph{Height}” and “\emph{Average Time}.” In the \emph{Height} column,
type in each height that you used. In the \emph{Average} Time column, type in the average time for each
run (or you can reference the cell in which you made that calculation).\\\\
See example below\\

\newpage

\begin{figure}[h!]
  \centerline{\includegraphics[scale=0.45]{resources/photo12.png}}
\end{figure}

Looking once again at the formula: $=(2 \; * \; (\text{first height cell}) / ((\text{first average time cell})^2))$, notice that
the “first height cell” here is C13, the first average time cell is D13, and the formula itself is in
E13. When you type this formula into the E13 cell, you can then copy and paste it into E14, E15
etc. The cell references will automatically be updated; you should not have to fix them.

\begin{itemize}
  \item For each ball, plot $|\Delta y|$ verses $\frac{t^2}{2}$ using the xy \emph{scatter}
    chart and insert a \emph{best-fit linear trendline} with its \underline{equation} to show the experimental
    $g$, which will be the slope of the trendline
  \item For each ball, calculate the percent difference between $g_{exp}$, the slope
    of your trendline, and the theoretical result $(g_{theoretical} = 9.80\frac{m}{s^2})$.
\end{itemize}
