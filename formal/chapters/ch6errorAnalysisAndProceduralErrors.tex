\chapter{Error Analysis and Procedural Errors}

\section{Front-End Error - RSS}

% \begin{equation}
%   \begin{split}
%   \\[-1ex]
%      & \Delta\sigma_{V,ij} = \\
%       &\begin{multlined}[0.9\linewidth]
%       \sqrt{\Delta\sigma_{x,ij}^2 + \Delta\sigma_{y,ij}^2 + \Delta\sigma_{z,ij}^2 - (\Delta\sigma_{x,ij} * \Delta\sigma_{y,ij} + \Delta\sigma_{y,ij} * \Delta\sigma_{z,ij} + \Delta\sigma_{z,ij} * \Delta\sigma_{x,ij})}\hdashrule[2.66ex]{1.8em}{0.4pt}{2pt} \\[0.5ex]
%        {\hdashrule[2.67ex]{1.8em}{0.4pt}{2pt}}\mkern -4mu\overline{\rule{0pt}{2.4ex}+ 3 * (\Delta\tau_{xy,ij}^2 + \Delta\tau_{yz,ij}^2 + \Delta\tau_{zx,ij}^2)}
%       \end{multlined}
%       \end{split}
%   \end{equation}
%   \end{fleqn}

%%%%%%%%%%%%%%%%% 3/8" Steel Ball RSS Below %%%%%%%%%%%%%%%%%

% TODO make this universal for all
\begin{fleqn}
\begin{equation*}
  \begin{split}
  \\[-1ex]
      & RSS = \\
      &\begin{multlined}[0.9\linewidth]
      \sqrt{\left( \frac{0.005m}{0.200m} \right)^2 + \left( \frac{0.005m}{0.400m} \right)^2 + \left( \frac{0.005m}{0.600m} \right)^2 + \left( \frac{0.005m}{0.800m} \right)^2 + \left( \frac{0.005m}{1.000m} \right)^2 + \left( \frac{0.00005s}{0.2025s} \right)^2 }\hdashrule[2.66ex]{1.8em}{0.4pt}{2pt} \\[0.5ex]
      {\hdashrule[2.67ex]{1.8em}{0.4pt}{2pt}}\mkern -4mu\overline{\rule{0pt}{2.4ex} + \left( \frac{0.00005s}{0.2868s} \right)^2 + \left( \frac{0.00005s}{0.3492s} \right)^2 + \left( \frac{0.00005s}{0.4040s} \right)^2 + \left( \frac{0.00005s}{0.4519s} \right)^2} \cdot 100\% = 3.0\% \\
      \end{multlined}
  \end{split}
\end{equation*}
\end{fleqn}

% TODO greatest contributor
\begin{equation*}
  RSS = \sqrt{\left( \frac{0.005m}{0.200m} \right)^2 + \left( \frac{0.00005s}{0.2025s} \right)^2} \cdot 100\% = 2.5\%
\end{equation*}

These values are chosen because they are the greatest contributors to error in 
this lab experiment

%%%%%%%%%%%%%%%%% 1/2" Steel Ball RSS Below %%%%%%%%%%%%%%%%%

%!!!!!!!!!!!!!!!!!!!!!!!!!!!!!!!!!!
% \begin{fleqn}
% \begin{equation*}
%   \begin{split}
%   \\[-1ex]
%       & \text{1/2" Steel Ball} \; RSS = \\
%       &\begin{multlined}[0.9\linewidth]
%       \sqrt{\left( \frac{0.005m}{0.200m} \right)^2 + \left( \frac{0.005m}{0.400m} \right)^2 + \left( \frac{0.005m}{0.600m} \right)^2 + \left( \frac{0.005m}{0.800m} \right)^2 + \left( \frac{0.005m}{1.000m} \right)^2 + \left( \frac{0.00005s}{0.2014} \right)^2 }\hdashrule[2.66ex]{1.8em}{0.4pt}{2pt} \\[0.5ex]
%        {\hdashrule[2.67ex]{1.8em}{0.4pt}{2pt}}\mkern -4mu\overline{\rule{0pt}{2.4ex} + \left( \frac{0.00005s}{0.2846s} \right)^2 + \left( \frac{0.00005s}{0.3485} \right)^2 + \left( \frac{0.00005s}{0.4033} \right)^2 + \left( \frac{0.00005s}{0.4509s} \right)^2} \cdot 100\% = 3.0\%
%       \end{multlined}
%   \end{split}
% \end{equation*}
% \end{fleqn}

% \begin{equation*}
%   \text{1/2" Steel Ball} \; RSS = \sqrt{\left( \frac{0.005m}{0.200m} \right)^2 + \left( \frac{0.00005s}{0.2014s} \right)^2} \cdot 100\% = 2.5\%
% \end{equation*}
%!!!!!!!!!!!!!!!!!!!!!!!!!!!!!!!!!!

\section{Back-End Error}

In this case, percent error should be used because this is an experimental
result being compared to the accepted known value of $g = 9.8\frac{m}{s^2}$
With the following percent error calculated, none of the data points were over 
the Front-End Error percentage. In general the calulated gravitational acceleration
was very close to the theoretical value of Earth's $g = 9.8\frac{m}{s^2}$ 

\newpage

\begin{equation*}
  3/8" \; ball: \hspace{1cm} \%error = \frac{|E - K|}{K} \cdot 100\%
\end{equation*}

\begin{equation*}
  = \frac{|9.8186\frac{m}{s^2} - 9.80\frac{m}{s^2}|}{9.80\frac{m}{s^2}} \cdot 100\% = 0.19\% \; (0.189\%)
\end{equation*}

\begin{equation*}
  1/2" \; ball: \hspace{1cm} \%error = \frac{|E - K|}{K} \cdot 100\%
\end{equation*}

\begin{equation*}
  = \frac{|9.8242\frac{m}{s^2} - 9.80\frac{m}{s^2}|}{9.80\frac{m}{s^2}} \cdot 100\% = 0.25\% \; (0.246\%)
\end{equation*}

\section{Potential Cause of Error}
There are a number of potential causes to the error presented in this lab 
experiment. The two most notable causes of error are as follows:
\begin{itemize}
  \item Miscalculated height - although we were aiming for 0.005m within each labeled
    drop height. There is possibilty that, that was not achieved for every drop due to changing
    heigts for different points of data collection. Additionally, it is possible, that 
    the surface the ball was dropped on was not perfectly level which could cause additional
    error in the experiemnt
  \item Timing limitations - the timing mechanism is only accurate a few decimal places\\
\end{itemize}
A third cause to error is the disregarded air resistance. Although it is stated 
that air resistance only has significant impact on large objects falling far distances,
air resistance could be a factor for some error.
    
