\chapter{Conclusion}

In conclusion, the purpose of this lab was to measure an experimental $g$ and comapre
that value with the accepted theoretical value of Earth's gravitational acceleration
$9.80\frac{m}{s^2}$. After collecting 25 points of data for each steel ball the 
$\frac{1}{2}"$ and $\frac{3}{8}"$ ball, it is clear that the theory of an object freely falling
will accelerate at this theoretical value. With both calculations of $g_{exp}$ being approximately
$9.82\frac{m}{s^2}$ and $9.81\frac{m}{s^2}$ respectively. \\\\
The $g_{exp}$ for the $\frac{1}{2}"$ ball was $9.8242\frac{m}{s^2}$ and\\\\
The $g_{exp}$ for the $\frac{3}{8}"$ ball was $9.8186\frac{m}{s^2}$\\\\
Though these $g_{exp}$'s are not exactly $9.80\frac{m}{s^2}$, the error allowed 
could account for this deviation from the theoretical value. With the RSS error being 
$3.0$\% and the greatest contributors RSS error being $2.5$\%\\\\
Furthermore, the backend error implementing \%error for the $\frac{1}{2}"$ ball was
$0.25$\% and the \%error for the $\frac{3}{8}"$ ball was $0.19$\%\\\\
Considering the purpose of the lab, the result, and factors of possible error, 
the goal of this lab was accomplished.
