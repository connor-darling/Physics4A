% \begingroup
% \let\clearpage\relax\chapter{Error Analysis and Procedural Errors}
\newgeometry{top=5mm, bottom=17mm}
\chapter{Error Analysis and Procedural Errors}

\section{Front-End Error - RSS}

For RSS, the greatest contributors are used. The greatest contributors are 
the lowest weighing mass configuration in $kg$, and the initial position of 
the spring in $m$.
  
  \begin{equation*}
    RSS = \sqrt{\left( \frac{0.0005m}{0.782m} \right)^2 
    + \left( \frac{0.0005kg}{1.036kg} \right)^2} \times 100\% = 0.08\% \; (0.080\%)
  \end{equation*}

\section{Back-End Error}

Percent difference was employed on this experiment because the experiemtnal value was 
being compared with a known actual value. The goal of the experiment was to compare (or
get the difference) between the two values. In general the experiemtnal spring constant $k_{exp}$
was close to the acutal spring constant value.

\begin{equation*}
  \%\text{difference}= \frac{|E_1 - E_2|}{\frac{E_1 + E_2}{2}} \times 100\% = 
  \frac{|276.52 - 278|}{\frac{276.52 + 278}{2}} \times 100\% = 0.53\% \; (0.534\%)
\end{equation*}

\section{Potential Cause of Error}

\noindent There are a number of potential causes to the error presented in this lab
experiment. The two most notable causes of error are as follows:

\begin{itemize}
  \item Miscalculated height - although we were aiming for within $0.0005m$ each time
        we measured the length the spring stretched, there is a possibility that this
        preciseness was not met for all eight configurations of mass. Additionally,
        it is also possible that the tabel our experiment took place on was not level,
        clouding an accurate measurement within $0.0005m$.
  
  \item Scale limitations - the scale weighing mechanism is only accurate a few
        decimal places.
  
\end{itemize}

\section{Error Analysis Questions}

\begin{itemize}
  \item Compare your experimental value(s) of $k_{exp}$ with the actual value(s) of $k$ for 
        your spring. (Spring $k$ is $278$ N/m.)
				
			\begin{center}
				\fbox{\begin{minipage}{23em}
								The actual constant value of the spring is $278$ N/m.\\
								The experimental value of the spring is $276.52$ N/m.\\
								The difference between the two values is $1.48$ N/m.\\
							\end{minipage}}
			\end{center}
	
  \item Do your results agree with Hooke's law (i.e. is $F$ directly proportional to 
        $x$)?

			\begin{center}
				\fbox{\begin{minipage}{26em}
								The results obtained from the experiment were not exact, but are close enough to 
								to agree with Hooke's law.
							\end{minipage}}
			\end{center}
\end{itemize}
\restoregeometry

% ! lol moment didnt even need to use all values for this RSS i could just use greatest
% ! constributors like i do above, but since i did it, its nice to have
% \begin{fleqn}
%   \begin{equation*}
%     \begin{split}
%     \\[-1ex]
%         & RSS = \\
%         &\begin{multlined}[0.9\linewidth]
%         \sqrt{\left( \frac{0.0005kg}{1.036kg} \right)^2 + \left( \frac{0.0005kg}{2.033kg} \right)^2 
%         + \left( \frac{0.0005kg}{4.033kg} \right)^2 + \left( \frac{0.0005kg}{6.032kg} \right)^2 
%         + \left( \frac{0.0005kg}{1.528kg} \right)^2 
%         + \left( \frac{0.0005kg}{3.527kg} \right)^2 }\hdashrule[4.45ex]{1.8em}{0.4pt}{2pt} \\[0.5ex]
%         {\hdashrule[4.15ex]{1.8em}{0.4pt}{2pt}}\mkern -4mu\overline{\rule{0pt}{2.4ex} 
%         + \left( \frac{0.0005kg}{5.526kg} \right)^2 + \left( \frac{0.0005kg}{2.530kg} \right)^2 
%         + \left( \frac{0.005m}{0.784m} \right)^2 + \left( \frac{0.005m}{0.803m} \right)^2 
%         + \left( \frac{0.005m}{0.876m} \right)^2 
%         + \left( \frac{0.005m}{0.951m} \right)^2 } \hdashrule[4.15ex]{1.8em}{0.4pt}{2pt} \\[0.5ex]
%         {\hdashrule[4.15ex]{1.8em}{0.4pt}{2pt}}\mkern -4mu\overline{\rule{0pt}{2.4ex} 
%         + \left( \frac{0.005m}{0.786m} \right)^2 + \left( \frac{0.005m}{0.860m} \right)^2 
%         + \left( \frac{0.005m}{0.933m} \right)^2 + \left( \frac{0.005m}{0.822m} \right)^2
%         + \left( \frac{0.005m}{0.782m} \right)^2} \times 100\% = CALC\% \\
%         \end{multlined}
%     \end{split}
%   \end{equation*}
%   \end{fleqn}