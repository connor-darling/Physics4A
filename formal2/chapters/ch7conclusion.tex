% \chapter{Conclusion}
\newgeometry{top=5mm, bottom=35mm}
\begingroup
\let\clearpage\relax\chapter{Conclusion}

In conclusion, the purpose of this lab was to measure an experimental spring 
constant $k_{exp}$ and compare it with the actual value of $278$ N/m. This was to be done 
by attaching a hanging spring and measuing the $x_i$ value of it. Then, attaching
8 different configurations of that weight to the spring and measure the $x_f$
value. After finding the $\Delta x$, the force $F$ in Nm could be found.
Once the force $F$ was calculated, it could be plotted versus the $\Delta x$,
and the slope of a best fit line was used to find the experimental $k_{exp}$.\\

\noindent Following this process, the $k_{exp}$ ended up being $276.52$ N/m. Though this is not
eaxctly $278$ N/m, the error allowed, could account for this deviation from the actual
value. The RSS error using greastest contributors ending up as $0.08\%$.\\

\noindent Furthermore, the backend error implementing \%difference was $0.53\%$.\\

\noindent Considering the purpose of the lab, the result, and factors of possible error;
the goal of this lab was accomplished.

\chapter{Suggestions for Improvement}
% \let\clearpage\relax\chapter{Suggestions for Improvement}
\endgroup

\noindent Due to the age and simplicity of this lab, the professors
of the Saddleback College Physics department have implemented almost all of the
improvements that could be made to this lab.\\

\noindent Despite this, if I were to improve the lab in any way, it would be to 
provide a much more indepth procedure section with an elaborate section on 
apparatus set up. With that, clear instructions of where and how to measure
should be made. If I did not have access to an example of my professors 
apparatus set up, I do not think my error would be as minimal as it was. Furthermore,
without the guidence of my professor, I would have taken incorrect measurements due
to the lack of clarity when it comes to the measurement in the procedure.\\