\chapter{Equations for Moment of Inertia Lab}
\section*{Theoretical Times}

\fbox{\begin{minipage}{20em}
  \begin{center}
    \noindent Theoretical time for solid cylinder at 5\degree
    \begin{equation*}
      \sqrt{ \frac{2(1 + 0.5) \cdot 1.000}{9.8 \sin (5)} } = 1.8741 \; sec
    \end{equation*}

    \noindent Theoretical time for hollow cylinder at 5\degree
    \begin{equation*}
      \sqrt{ \frac{2(1 + 1) \cdot 1.000}{9.8 \sin (5)} } = 2.1641 \; sec
    \end{equation*}

    \noindent Theoretical time for sphere at 5\degree
    \begin{equation*}
      \sqrt{ \frac{2(1 + \frac{2}{5}) \cdot 1.000}{9.8 \sin (5)} } = 1.8106 \; sec
    \end{equation*}
  \end{center}
\end{minipage}}\\

\fbox{\begin{minipage}{20em}
  \begin{center}
    \noindent Theoretical time for solid cylinder at 10\degree
    \begin{equation*}
      \sqrt{ \frac{2(1 + 0.5) \cdot 1.000}{9.8 \sin (10)} } = 1.3330 \; sec
    \end{equation*}

    \noindent Theoretical time for hollow cylinder at 10\degree
    \begin{equation*}
      \sqrt{ \frac{2(1 + 1) \cdot 1.000}{9.8 \sin (10)} } = 1.5392 \; sec
    \end{equation*}

    \noindent Theoretical time for sphere at 10\degree
    \begin{equation*}
      \sqrt{ \frac{2(1 + \frac{2}{5}) \cdot 1.000}{9.8 \sin (10)} } = 1.2878 \; sec
    \end{equation*}
  \end{center}
\end{minipage}}

\section*{RSS Error}

\begin{center}
  \noindent Greatest contributors are used to calculate RSS error
  \begin{equation*}
      \text{RSS} = \sqrt{ \left( \frac{0.00005 \; sec}{1.2037 \; sec} \right)^2 
      + \left( \frac{0.0005 m}{1.000 m} \right)^2
      + \left( \frac{0.00005 m}{0.0255 m} \right)^2} \times 100\% = 1.02\%
  \end{equation*}
\end{center}

\section*{Back-End Error}

\begin{center}
  \noindent Percent error should be used because we are calculating a theoretical
  value and comparing an experimental value with that number rather than comparing
  two unknown experimental values.\\
  \begin{equation*}
      \% \text{error} = \frac{|E - K|}{K} \times 100 \%
  \end{equation*}

  % \begin{equation*}
  %   \%\text{difference}= \frac{|E_1 - E_2|}{\frac{E_1 + E_2}{2}} \times 100\% = 
  %   \frac{|276.52 - 278|}{\frac{276.52 + 278}{2}} \times 100\% = 0.53\% \; (0.534\%)
  % \end{equation*}

  \noindent Where $E =$ experimental value and $K =$ theoretical value.

  % ! 5 degrees
  \begin{equation*}
      \% \text{error for solid cylinder at 5\degree} 
      = \frac{|1.9279 \; sec - 1.8741 \; sec|}{1.8741 \; sec} \times 100\% = 2.87\%
  \end{equation*}\\

  \begin{equation*}
      \% \text{error for hollow cylinder at 5\degree} 
      = \frac{|\; 2.1946 sec - 2.1641 \; sec|}{2.1641 \; sec} \times 100\% = 1.41\%
  \end{equation*}\\

  \begin{equation*}
      \% \text{error for sphere at 5\degree} 
      = \frac{|1.8141 \; sec - 1.8106 \; sec|}{1.8106 \; sec} \times 100\% = 0.19\%
  \end{equation*}\\
  
  % ! 10 degrees
  \begin{equation*}
      \% \text{error for solid cylinder at 10\degree} 
      = \frac{|1.2564 \; sec - 1.3330 \; sec|}{1.3330 \; sec} \times 100\% = 5.75\%
  \end{equation*}\\

  \begin{equation*}
      \% \text{error for hollow cylinder at 10\degree} 
      = \frac{|1.4367 \; sec - 1.5392 \; sec|}{1.5392 \; sec} \times 100\% = 6.66\%
  \end{equation*}\\

  \begin{equation*}
      \% \text{error for sphere at 10\degree} 
      = \frac{|1.2151 \; sec - 1.2878 \; sec|}{1.2878 \; sec} \times 100\% = 5.65\%
  \end{equation*}\\
\end{center}

\section{More Analysis (Q4)}

\noindent Using:
\begin{equation*}
  \Delta t = \frac{1}{R_2} \sqrt{ \frac {D(3R_2^2 + R_1^2)}{g \sin \theta} }
\end{equation*}

\section*{Calculated $\Delta t$ for hollow cylinder}

\noindent Trial using $\theta = 5\degree$:\\
\begin{equation*}
  \Delta t = \frac{1}{0.0285} \sqrt{ \frac {1.000(3 \cdot (0.0255)^2 + 0.0285^2)}{9.8 \sin (5)} } = 1.9957 \; sec
\end{equation*}

\noindent Trial using $\theta = 10\degree$:

\begin{equation*}
  \Delta t = \frac{1}{0.0285} \sqrt{ \frac {1.000(3 \cdot (0.0255)^2 + 0.0285^2)}{9.8 \sin (10)} } = 1.4138 \; sec
\end{equation*}

\begin{center}
  \begin{tabular} { |p{2cm}|p{2cm}|p{2cm}|p{2cm}|p{2cm}| }
    \hline
    \multicolumn{5}{|c|}{ Comparing Values } \\
    \hline 
    \centering & Theoretical using C as 1 & Theoretical using an $R_1$ \& $R_2$ &  Experimental & \%error \\ 
    \hline
    $\theta = 5\degree$ & $2.1641 \; sec$ & $1.9957 \; sec$ & $2.1946 \; sec$ & $1.41\%$ \\
    \hline
    $\theta = 10\degree$ & $1.5392 \; sec$ & $1.4138 \; sec$ & $1.4367 \; sec$ & $6.66\%$ \\
    \hline
  \end{tabular}
\end{center}

Analyzing these results show that using $C = 1$ was closer to the experimental value than
using two radii for the $5\degree$ trial. However, for the $10\degree$ trial, the result 
is the opposite. Factoring in the \%error, our error was much higher for the $10\degree$ trial.
This tells me that with minimal error, using $C = 1$ is more accurate is this context.

\section*{4 - Procedural Errors and Improvements}

Some of the contributing factors the errors in this lab are, inclometer limitations, calliper limitations,
and photogate limitations. Though the goal of these devices are to attempt to give the best reading possible,
the limitation to a specific decimal places allows room for error, which is why we allowed our data to be +-
a certain amount. A big contributor to error could be releasing the ball. It is difficult to release the ball exactly 
straight which not doing so would default the goal of the lab leading to error. Another factor for error is the 
initial velocity of the rolled object. It is difficult to realse the object from rest with out triggering 
the photogate timer, which also could lead to error. I would suggest that students or lab participants use something
to help keep the object to roll in a straight line without disrupting its velocity.

\chapter{Conservation of Energy and Momentum}
\section*{RSS Error}

\begin{center}
  \noindent Greatest contributors are used to calculate RSS error
  \begin{equation*}
      \text{RSS\% Error} = \sqrt{ \left( \frac{0.00005kg}{0.2156kg} \right)^2 
      + \left( \frac{0.005 m/s}{0.18m/s} \right)^2}
      \times 100\% = 2.78\%
  \end{equation*}
\end{center}

\section*{Back-End Error \% Difference}

\begin{center}
  \begin{equation*}
    \text{\% Difference} = \frac{|E_1 - E_2|}{\frac{(E_1 + E_2)}{2}} \times 100\% 
    = \frac{|0.11 \frac{kgm}{s} - 0.09 \frac{kgm}{s}|}{\frac{(0.11 \frac{kgm}{s}+ 0.09 \frac{kgm}{s})}{2}} \times 100\% = 19.50\%
  \end{equation*}
\end{center}

\noindent Percent difference should be used to calculate error because we are comparing the 
initial momentum versus the final momentum. This explicit calculation was done for Trial\#1's
momentum values. This was an elastic collision where theoretically the inital momentum and
final momentum should be the same, but as shown this is not the case. This shows that there was
definitely error involved.

\newpage

\section*{4 - Procedural Errors and Improvements}

\noindent \textbf{Procedural Errors}
\begin{itemize}
  \item In regards to errors, there are factors that limit the most precise and 
        accurate readings. One of these factors are the software limiting us to 
        specific decimal places
  \item Additionally, the scale is limited to four places past the decimal point
        allowing for error.
  \item Another factor is the air track we used to simulate a frictionless surface
        for the carts to glide upon. Of course it would be impossible to get a 
        perfectly frictionless surface in the context of this lab, but without that
        preciseness, this lab is error prone.
  \item Finally, in the software itself of data studio, a dfferent selection of 
        data would yield different results. This could lead not only errors, but
        a biased selection of data.
  \item$\star$ It is noted that our error was awful for this lab, we re-ran trials 
               multiple times included with your supervision, but still we were 
               getting extremely high error. You advised us to not use this lab
               as a formal write up.
\end{itemize}

\noindent \textbf{Improvements}
\begin{itemize}
  \item If I could reccomend something to be improved, it would be a better printed
        procedural section. Some parts of the print were messed up make it difficult
        to read and understand.
\end{itemize}

\chapter*{\centering Rotational Inertia of a Point Mass Lab}
\begin{center}
  \Huge ``Freebie''
\end{center}

\chapter*{\centering Centripetal Force Lab}
\begin{center}
  \Huge ``Freebie''
\end{center}


\chapter*{Simple Pendulum Lab}

\section*{RSS Error}

\begin{center}
  \noindent Greatest contributors are used to calculate RSS error
  \begin{equation*}
      \text{RSS\% Error} = \sqrt{ \left( \frac{0.00005kg}{0.0104kg} \right)^2 
      + \left( \frac{0.0005 m}{0.193 m} \right)^2 + \left( \frac{0.00005 s}{0.8991 s} \right)^2}
      \times 100\% = 0.55\%
  \end{equation*}
\end{center}

\section*{Back-End Error \% Difference}

\begin{center}
  \begin{equation*}
    \text{\% Difference} = \frac{|E_1 - E_2|}{\frac{(E_1 + E_2)}{2}} \times 100\% 
    = \frac{|1.3836 s - 1.3978 s|}{\frac{(1.3836 s + 1.3978 s)}{2}} \times 100\% = 1.01\%
  \end{equation*}
\end{center}

\noindent Percent difference should be used to calculate error between the times of different mass, amplitudes, and lengths, 
because we are calculating multiple experimental values and comparing their differences. 

\section*{Back-End Error \% Error}

  \begin{equation*}
      \% \text{error} = \frac{|E - K|}{K} \times 100 \% = \frac{|9.8 m/s^2 - 9.6369 m/s^2|}{9.6369 m/s^2} = 1.69\%
  \end{equation*}

\noindent Percent error should be used to calculate error between the $g_{experimental}$
and the $g_{theoretical}$, because $g$ is a scientifically known and accepted value. That being
said, we want to know how off we were when attempted to acieve this value experimentally.

\newpage 

\section*{4 - Procedural Errors and Improvements}

\noindent \textbf{Procedural Errors}
\begin{itemize}
  \item In regards to errors, there are factors that limit the most precise and 
        accurate readings. One of these factors are the timer limiting the time
        to four decimal places.
  \item Additionally, the scale is limited to four places past the decimal point
        allowing for error.
  \item Another factor is actually measuring the length of the string. We had to
        rely on a meter stick and were only accurate within three decimal places.
        We actually ran into an issue where our initial data was skewed by a bad 
        reading of the length of the string, leading us to have our gravitational
        acceleration as around $35 \frac{m}{s^2}$. Once we re-measured, we were able to get
        a much more accurate representaion of $g_{experimental}$.
\end{itemize}

\noindent \textbf{Improvements}
\begin{itemize}
  \item If I could recommend something to be improved, it would to be much more
        careful when measuring the length of the string to the center of the mass.
        My partner and I really suffered the consequences due to this negligance.
\end{itemize}

\chapter*{Archimede's Principle Lab}

\section*{RSS Error}

\begin{center}
  \noindent Greatest contributors are used to calculate RSS error.
  \begin{equation*}
      \text{RSS\% Error} = \sqrt{ \left( \frac{0.00005 kg}{0.0505 kg} \right)^2 
      + \left( \frac{0.00005 m}{0.0355 m} \right)^2}
      \times 100\% = 0.17\%
  \end{equation*}
\end{center}

\section*{Back-End Error \% Error}

  \begin{equation*}
      \% \text{error} = \frac{|E - K|}{K} \times 100 \% 
  \end{equation*}

  \begin{equation*}
      \% \text{error}_{Small \; Can} = \frac{|0.1536 \; kg - 0.1592 \; kg|}{0.1592 \; kg} = 3.52\%
  \end{equation*}
  % where $E = 0.1536 \; kg$ and $K = 0.1592 \; kg$.

  \begin{equation*}
      \% \text{error}_{Large \; Can} = \frac{|0.2583 \; kg - 0.2761 \; kg|}{0.2761 \; kg} = 6.45\%
  \end{equation*}

  \noindent I decided to use percent error, because instead of comparing two experimental values,
  we are comparing a calculated value with a measured value. That being the value derived from 
  Archimede's Principle, and the actually value we measured when placing mass into the cylinder.
  However, the Theory section of the lab has noted that ``the experimental mass needed to sink the
  can will be the calculated mass and the theoretical mass will be the measured one.''

  \newpage

  \section*{4 - Procedural Errors and Improvements}

    \noindent \textbf{Procedural Errors}
    \begin{itemize}
      \item In regards to errors, there are factors that limit the most precise and 
            accurate readings. One of these factors are the callipers limiting measurments
            to four decimal places.
      \item Additionally, the scale is limited to four places past the decimal point
            allowing for error.
      \item Another factor that could have led to error is the force used to place a mass
            inside the cans while measuring how much mass it took to sink. It is entirely
            possible that we used too much force which could have led to the can filling with water
            and sinling prematurely.
    \end{itemize}

    \noindent \textbf{Improvements}
    \begin{itemize}
      \item If I could recommend something to be improved, it would to be much more
            careful when putting the mass inside the can. Due to my partner and I taking
            a lot of time with another lab, we were really rushed with time. Due to this
            I would make sure we ensure there is adequate time to finish both labs.
    \end{itemize}

    \chapter*{Random}

    \section*{Taylor Series}
    $$f(x) = \sum_{n=0}^{\infty} \frac{f^{(n)}(a)}{n!} (x-a)^n$$

    \noindent Here, $f^{(n)}(a)$ is the $n$th derivative of $f(x)$ evaluated at $a$, and $n!$ is the factorial of $n$. 
    The series is centered at $a$, and the terms in the series involve raising $x-a$ to increasingly higher powers.

    \section*{Schrödinger Equation}

    \noindent The Schrödinger equation is a fundamental equation of quantum mechanics 
    that describes how a wave function changes over time:

    $$i\hbar\frac{\partial}{\partial t} \Psi(\mathbf{r}, t) = \hat{H} \Psi(\mathbf{r}, t)$$

    In this equation, $\Psi(\mathbf{r}, t)$ is the wave function, $i$ is the imaginary unit, 
    $\hbar$ is the reduced Planck constant, and $\hat{H}$ is the Hamiltonian operator.

    \section*{Heisenberg Uncertainty Principle}

    The Heisenberg uncertainty principle is a fundamental principle of quantum mechanics 
    that states that the position and momentum of a particle cannot be measured with arbitrary precision:

    $$\Delta x \Delta p \ge \frac{\hbar}{2}$$

    In this equation, $\Delta x$ and $\Delta p$ are the uncertainties in the position 
    and momentum of the particle, respectively, and $\hbar$ is the reduced Planck constant.

    \section*{The Pauli Exclusion Principle}

    The Pauli exclusion principle is a fundamental principle of quantum mechanics 
    that states that no two particles can occupy the same quantum state simultaneously:

    $$\text{No two electrons can have the same set of quantum numbers}$$

    \section*{The Standard Model Lagrangian}

    $$\mathcal{L} = \sum_{i=1}^3\left[\left(\bar{\psi}i, i\gamma^{\mu}\partial{\mu}\psi_i\right) + 
    \left(m_i \bar{\psi}i\psi_i\right)\right] - \frac{1}{4}F^{\mu\nu}F{\mu\nu} + 
    \frac{1}{2}\left(D_{\mu}H\right)^{\dagger}\left(D^{\mu}H\right) - V\left(H\right)$$

    where $F^{\mu\nu}$ is the field strength tensor for the gauge fields, $\psi_i$ are the 
    three generations of fermions (quarks and leptons), $H$ is the Higgs field, $m_i$ are the fermion masses, 
    $V\left(H\right)$ is the Higgs potential, and $D_{\mu}$ is the covariant derivative.



    \chapter*{Greek Symbols}

      $$\alpha \text{ alpha and Alpha } A$$
      $$\beta \text{ beta and Beta } B$$
      $$\gamma \text{ gamma and Gamma } \Gamma$$
      $$\delta \text{ delta and Delta } \Delta$$
      $$\epsilon \text{ epsilon and Epsilon } E$$
      $$\zeta \text{ zeta and Zeta } Z$$
      $$\eta \text{ eta and Eta } H$$
      $$\theta \text{ theta and Theta } \Theta$$
      $$\iota \text{ iota and Iota } I$$
      $$\kappa \text{ kappa and Kappa } K$$
      $$\lambda \text{ lambda and Lambda } \Lambda$$
      $$\mu \text{ mu and Mu } M$$
      $$\nu \text{ nu and Nu } N$$
      $$\xi \text{ xi and Xi } \Xi$$
      $$\pi \text{ pi and Pi } \Pi$$
      $$\rho \text{ rho and Rho } P$$
      $$\sigma \text{ sigma and Sigma } \Sigma$$
      $$\tau \text{ tau and Tau } T$$
      $$\upsilon \text{ upsilon and Upsilon } \Upsilon$$
      $$\phi \text{ phi and Phi } \Phi$$
      %$$\chi \text{ chi and Chi } \Chi$$
      $$\psi \text{ psi and Psi } \Psi$$
      $$\omega \text{ omega and Omega } \Omega$$