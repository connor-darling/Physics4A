\chapter*{Phys 4A Final Exam Cheat Sheet Equations}

Following this page lie equations I will be using on my cheat sheet
for my final exam.

%% ! 1-D Kinematics
\chapter*{1-D Kinematics}

\fbox{\begin{minipage}{0.35\textwidth}
    \textbf{Horizontal Equations of Motion}
      \begin{equation*}
        v_f = v_0 + at
      \end{equation*}
      \begin{equation*}
        x_f = x_0 + v_0t + \frac{1}{2}at^2
      \end{equation*}
      \begin{equation*}
        x_f = x_0 + \frac{1}{2}[v_0 + v_f]t
      \end{equation*}
      \begin{equation*}
        v^2_f = v^2_0 + 2a[x_f - x_0]
      \end{equation*}
    \textbf{Vertical Equations of Motion}
      \begin{equation*}
        v_f = v_0 - gt
      \end{equation*}
      \begin{equation*}
        y_f = y_0 + v_0t - \frac{1}{2}gt^2
      \end{equation*}
      \begin{equation*}
        y_f = y_0 + \frac{1}{2}[v_0 + v_f]t
      \end{equation*}
      \begin{equation*}
        v^2_f = v^2_0 - 2g[y_f - y_0]
      \end{equation*}
  \vspace{0.5cm}
  \end{minipage}}




%% ! Torque 
\chapter*{Torque}

\fbox{\begin{minipage}{0.6\textwidth}
    \textbf{Torque}

    \begin{equation*}
      \tau = \vec{r} \times \vec{F}
      % \phantom{\hspace{12cm}}
    \end{equation*}

    \begin{equation*}
      \tau = I\alpha
      % \phantom{\hspace{12cm}}
    \end{equation*}

    \begin{itemize}
      \item Need to know direction of torque
      \item Use right hand rule (fingers curve direction of rotation, torque is where thumb points)
      \item $\omega \; \& \; \alpha$ are the basics
      \item Torque is orthogonal to rotation
      \item Torque is the function of moment arm from origin
    \end{itemize}

  \vspace{0.5cm}
  \end{minipage}}


%% ! Gravitation
\chapter*{Gravitation}

% \section*{Kepler's 3rd Law}
%   \begin{equation*}
%     T^2 = \left(\frac{4\pi^2}{GM_s}\right)r^3
%     \phantom{\hspace{10cm}}
%   \end{equation*}


\fbox{\begin{minipage}{0.6\textwidth}
    \textbf{Gravitation}\\

    \textbf{Kepler's 3rd Law}

    \begin{equation*}
      |\vec{F}| = G \frac{M_s M_p}{r^2} = \frac{M_p v^2}{r}
    \end{equation*}

    \begin{equation*}
      \frac{GM_s}{r^2} = \frac{\left(\frac{2\pi r}{T}\right)^2}{r} \text{, where } v = \frac{2\pi r}{T}
    \end{equation*}

    \begin{equation*}
      T^2 = \left(\frac{4\pi^2}{GM_s}\right)r^3
      % \phantom{\hspace{10cm}}
    \end{equation*}


  \vspace{0.5cm}
  \end{minipage}}

%% !Fluids
\chapter*{Fluids}
\section*{Continuity Equation}
  For incompressible flow $(\rho_1 = \rho_2)$:\\\\
    $A_1 v_1 = A_2 v_2 = \text{constant}$\\

  \noindent consequences: small tube, high speed\\
  \hspace*{21.5mm} large tube, low speed\\

  \begin{equation*}
      Av = \left[ \frac{volume}{time} \right] = \text{volume flux = flow rate}
    \phantom{\hspace{10cm}}
  \end{equation*}

  \noindent ``what goes in must come out''

  \section*{Bernoulli's Equation}
  \begin{equation*}
      P_1 + \frac{1}{2} \rho v^2_1 + \rho g y_1 = P_2 + \frac{1}{2} \rho v^2_2 + \rho g y_2
    \phantom{\hspace{10cm}}
  \end{equation*}

  \hspace*{5mm} \noindent \textcolor{orange}{\bf{OR}}

  \begin{equation*}
      P + \frac{1}{2} \rho v^2 + \rho g y = \text{constant}
    \phantom{\hspace{12cm}}
  \end{equation*}

  \hspace*{5mm} \noindent \textcolor{orange}{\textbf{OR} when fluid is at rest, $v_1 = v_2 = 0$}

  \begin{equation*}
      (P_1 - P_2) = \rho gy_2 - \rho gy_1 = \rho g(y_2 - y_1) = \rho gh
    \phantom{\hspace{12cm}}
  \end{equation*}

  \begin{equation*}
      P_1 + \rho gy_1 = P_2 + \rho gy_2
    \phantom{\hspace{12cm}}
  \end{equation*}

  \begin{equation*}
      P_1 = P_2 + \rho g(y_2 - y_1)
    \phantom{\hspace{12cm}}
  \end{equation*}

%% ! Oscillations
\chapter*{Oscillations}

\section*{Simple Harmonic Motion}

\begin{minipage}{0.5\textwidth}
    For a horizontal spring:
    \begin{proof}
      \begin{equation*}
        F = -kx = ma
      \end{equation*}

      \begin{equation*}
        a = - \frac{k}{m}x
      \end{equation*}

      \[\blacksquare\]
    \end{proof}

  \vspace{0.5cm}
  \end{minipage}
  \begin{minipage}{0.5\textwidth}
    For a vertical spring:

    \begin{proof}
      \begin{equation*}
        F = mg - ky = ma \rightarrow -ky = m (a - g) = ma'
      \end{equation*}

        \textcolor{orange}{Where $(a - g) = a'$}

      \begin{equation*}
        a' = - \frac{k}{m}y
      \end{equation*}
      \[\blacksquare\]
    \end{proof}
  \end{minipage}

  \begin{equation*}
    \frac{d^2x}{dt^2} = - \frac{k}{m}x = - \omega^2x
  \end{equation*}

  \begin{equation*}
    {\text{Where } \omega^2 = \frac{k}{m}}
  \end{equation*}

  \newpage

\fbox{\begin{minipage}{0.5\textwidth}
    \textbf{Definition of Terms}
    \begin{equation*}
      x(t) = A\cos(\omega t+\delta)
    \end{equation*}

    \noindent $x(t)$: Displacement at time $t$\\
    \noindent $A$: Amplitude\\
    \noindent $\omega$: Angular Frequency\\
    \noindent $t$: Time\\
    \noindent $\delta$: Phase Constant\\
    \noindent $(\omega t+\delta)$: Phase\\

    \noindent The period, $T$, is the time for the particle\\
    \noindent to go through one full cycle of its motion.\\
    \noindent The phase increases by $2\pi$ in a time $T$\\

    \begin{equation*}
      \omega t + \delta + 2\pi = \omega (t + T) + \delta
      % \phantom{\hspace{12cm}}
    \end{equation*}

    \begin{equation*}
      \hspace*{0mm} 2\pi = \omega T
    \end{equation*}

    \begin{equation*}
      \hspace*{0mm} T = \frac{2\pi}{\omega} = 2\pi \sqrt{\frac{m}{k}}
    \end{equation*}

    \begin{equation*}
      \text{The freq. is inverse of period } \(\freq\) = \frac{1}{T} 
       = \frac{\omega}{2\pi} = \frac{1}{2\pi}\sqrt{\frac{k}{m}}
    \end{equation*}

    \begin{equation*}
      \text{rearranging } \omega = 2\pi \(\freq\) = \frac{2\pi}{T}
    \end{equation*}

  \vspace{0.5cm}
  \end{minipage}}

%!

\fbox{\begin{minipage}{0.5\textwidth}
    \textbf{Pendulum}
    \noindent The simple pendulum is another mechanical system that exhibits
    periodic, oscillatory motion
    \begin{equation*}
      \frac{d^2\theta}{dt^2} = -\frac{g}{L}\theta 
      \hspace*{4mm} {\text{which is similar to  } \frac{d^2x}{dt^2} = -\omega^2x}
    \end{equation*}

    \begin{equation*}
      \omega = \sqrt{\frac{g}{L}}
      \hspace*{4mm} {\text{and  } T = 2\pi \sqrt{\frac{L}{g}}}
    \end{equation*}
  \vspace{0.5cm}
  \end{minipage}}

\fbox{\begin{minipage}{0.5\textwidth}
    \textbf{Physical Pendulum}
    \noindent A physicsal pendulum consists of any rigid body suspended from
    a fixed axis that does not pass through the body's center of mass
    \begin{equation*}
      \frac{d^2\theta}{dt^2} = -\frac{mgd}{I}\theta 
    \end{equation*}

    \begin{equation*}
      \omega = \sqrt{\frac{mgd}{I}}
      \hspace*{4mm} {\text{and  } T = 2\pi \sqrt{\frac{I}{mgd}}}
    \end{equation*}
  \vspace{0.5cm}
  \end{minipage}}

  \fbox{\begin{minipage}{0.5\textwidth}
    \textbf{What About A \& $\delta$}

    \begin{equation*}
      x(t) &= A\cos(\omega t+\delta) \rightarrow x_0 = A\cos\delta
      \phantom{\hspace{12cm}}
    \end{equation*}

    \begin{equation*}
      v(t) = -A\omega \sin(\omega t+\delta) \rightarrow v_0 = -A\omega \sin\delta
      \phantom{\hspace{12cm}}
    \end{equation*}

    \noindent Dividing these EQ's

    \begin{equation*}
      \frac{v_0}{x_0} = -\omega \tan \delta
      \phantom{\hspace{12cm}}
    \end{equation*}

    \begin{equation*}
      \tan\delta = - \frac{v_0}{\omega x_0}
      \phantom{\hspace{12cm}}
    \end{equation*}

    \noindent Taking sum of squares:

    \begin{equation*}
      x^2_0 + \left(\frac{v_0}{\omega}\right)^2 = A^2(\cos\delta)^2
      = A^2[(\sin\delta)^2 + (\cos\delta)^2]
      \phantom{\hspace{12cm}}
    \end{equation*}

    \begin{equation*}
      A = \sqrt{x^2_0 + \left(\frac{v_0}{\omega}\right)^2}
      \phantom{\hspace{12cm}}
    \end{equation*}
    
  \end{minipage}}
