\documentclass{report}
\usepackage{amsmath}
\usepackage{amssymb}
\usepackage{graphicx}
%\usepackage[showframe]{geometry}
\usepackage{geometry}
\usepackage{mathtools, nccmath}
\usepackage{dashrule}
\usepackage[utf8]{inputenc}
\usepackage{changepage}
\usepackage{gensymb}

\title{\Huge Equations Live Here\\
{\huge Physics 4A}}
\author{
  \Large Connor Darling\\\\
}
\date{December 2022}

\begin{document}
\maketitle
\tableofcontents

\chapter{Equations for Moment of Inertia Lab}
\section*{Theoretical Times}

\fbox{\begin{minipage}{20em}
  \begin{center}
    \noindent Theoretical time for solid cylinder at 5\degree
    \begin{equation*}
      \sqrt{ \frac{2(1 + 0.5) \cdot 1.000}{9.8 \sin (5)} } = 1.8741 \; sec
    \end{equation*}

    \noindent Theoretical time for hollow cylinder at 5\degree
    \begin{equation*}
      \sqrt{ \frac{2(1 + 1) \cdot 1.000}{9.8 \sin (5)} } = 2.1641 \; sec
    \end{equation*}

    \noindent Theoretical time for sphere at 5\degree
    \begin{equation*}
      \sqrt{ \frac{2(1 + \frac{2}{5}) \cdot 1.000}{9.8 \sin (5)} } = 1.8106 \; sec
    \end{equation*}
  \end{center}
\end{minipage}}\\

\fbox{\begin{minipage}{20em}
  \begin{center}
    \noindent Theoretical time for solid cylinder at 10\degree
    \begin{equation*}
      \sqrt{ \frac{2(1 + 0.5) \cdot 1.000}{9.8 \sin (10)} } = 1.3330 \; sec
    \end{equation*}

    \noindent Theoretical time for hollow cylinder at 10\degree
    \begin{equation*}
      \sqrt{ \frac{2(1 + 1) \cdot 1.000}{9.8 \sin (10)} } = 1.5392 \; sec
    \end{equation*}

    \noindent Theoretical time for sphere at 10\degree
    \begin{equation*}
      \sqrt{ \frac{2(1 + \frac{2}{5}) \cdot 1.000}{9.8 \sin (10)} } = 1.2878 \; sec
    \end{equation*}
  \end{center}
\end{minipage}}

\section*{RSS Error}

\begin{center}
  \noindent Greatest contributors are used to calculate RSS error
  \begin{equation*}
      \text{RSS} = \sqrt{ \left( \frac{0.00005 \; sec}{1.2037 \; sec} \right)^2 
      + \left( \frac{0.0005 m}{1.000 m} \right)^2
      + \left( \frac{0.00005 m}{0.0255 m} \right)^2} \times 100\% = 1.02\%
  \end{equation*}
\end{center}

\section*{Back-End Error}

\begin{center}
  \noindent Percent error should be used because we are calculating a theoretical
  value and comparing an experimental value with that number rather than comparing
  two unknown experimental values.\\
  \begin{equation*}
      \% \text{error} = \frac{|E - K|}{K} \times 100 \%
  \end{equation*}

  % \begin{equation*}
  %   \%\text{difference}= \frac{|E_1 - E_2|}{\frac{E_1 + E_2}{2}} \times 100\% = 
  %   \frac{|276.52 - 278|}{\frac{276.52 + 278}{2}} \times 100\% = 0.53\% \; (0.534\%)
  % \end{equation*}

  \noindent Where $E =$ experimental value and $K =$ theoretical value.

  % ! 5 degrees
  \begin{equation*}
      \% \text{error for solid cylinder at 5\degree} 
      = \frac{|1.9279 \; sec - 1.8741 \; sec|}{1.8741 \; sec} \times 100\% = 2.87\%
  \end{equation*}\\

  \begin{equation*}
      \% \text{error for hollow cylinder at 5\degree} 
      = \frac{|\; 2.1946 sec - 2.1641 \; sec|}{2.1641 \; sec} \times 100\% = 1.41\%
  \end{equation*}\\

  \begin{equation*}
      \% \text{error for sphere at 5\degree} 
      = \frac{|1.8141 \; sec - 1.8106 \; sec|}{1.8106 \; sec} \times 100\% = 0.19\%
  \end{equation*}\\
  
  % ! 10 degrees
  \begin{equation*}
      \% \text{error for solid cylinder at 10\degree} 
      = \frac{|1.2564 \; sec - 1.3330 \; sec|}{1.3330 \; sec} \times 100\% = 5.75\%
  \end{equation*}\\

  \begin{equation*}
      \% \text{error for hollow cylinder at 10\degree} 
      = \frac{|1.4367 \; sec - 1.5392 \; sec|}{1.5392 \; sec} \times 100\% = 6.66\%
  \end{equation*}\\

  \begin{equation*}
      \% \text{error for sphere at 10\degree} 
      = \frac{|1.2151 \; sec - 1.2878 \; sec|}{1.2878 \; sec} \times 100\% = 5.65\%
  \end{equation*}\\
\end{center}

\section{More Analysis (Q4)}

\noindent Using:
\begin{equation*}
  \Delta t = \frac{1}{R_2} \sqrt{ \frac {D(3R_2^2 + R_1^2)}{g \sin \theta} }
\end{equation*}

\section*{Calculated $\Delta t$ for hollow cylinder}

\noindent Trial using $\theta = 5\degree$:\\
\begin{equation*}
  \Delta t = \frac{1}{0.0285} \sqrt{ \frac {1.000(3 \cdot (0.0255)^2 + 0.0285^2)}{9.8 \sin (5)} } = 1.9957 \; sec
\end{equation*}

\noindent Trial using $\theta = 10\degree$:

\begin{equation*}
  \Delta t = \frac{1}{0.0285} \sqrt{ \frac {1.000(3 \cdot (0.0255)^2 + 0.0285^2)}{9.8 \sin (10)} } = 1.4138 \; sec
\end{equation*}

\begin{center}
  \begin{tabular} { |p{2cm}|p{2cm}|p{2cm}|p{2cm}|p{2cm}| }
    \hline
    \multicolumn{5}{|c|}{ Comparing Values } \\
    \hline 
    \centering & Theoretical using C as 1 & Theoretical using an $R_1$ \& $R_2$ &  Experimental & \%error \\ 
    \hline
    $\theta = 5\degree$ & $2.1641 \; sec$ & $1.9957 \; sec$ & $2.1946 \; sec$ & $1.41\%$ \\
    \hline
    $\theta = 10\degree$ & $1.5392 \; sec$ & $1.4138 \; sec$ & $1.4367 \; sec$ & $6.66\%$ \\
    \hline
  \end{tabular}
\end{center}

Analyzing these results show that using $C = 1$ was closer to the experimental value than
using two radii for the $5\degree$ trial. However, for the $10\degree$ trial, the result 
is the opposite. Factoring in the \%error, our error was much higher for the $10\degree$ trial.
This tells me that with minimal error, using $C = 1$ is more accurate is this context.

\section*{4 - Procedural Errors and Improvements}

Some of the contributing factors the errors in this lab are, inclometer limitations, calliper limitations,
and photogate limitations. Though the goal of these devices are to attempt to give the best reading possible,
the limitation to a specific decimal places allows room for error, which is why we allowed our data to be +-
a certain amount. A big contributor to error could be releasing the ball. It is difficult to release the ball exactly 
straight which not doing so would default the goal of the lab leading to error. Another factor for error is the 
initial velocity of the rolled object. It is difficult to realse the object from rest with out triggering 
the photogate timer, which also could lead to error. I would suggest that students or lab participants use something
to help keep the object to roll in a straight line without disrupting its velocity.

% \newgeometry{top=5mm, bottom=20mm}
\chapter{Conservation of Linear Momentum \& Kinetic Energy}

\begin{center}
  \begin{tabular} { |p{0.75cm}|p{1.4cm}|p{1.45cm}|p{1.4cm}|p{1.45cm}|p{2cm}|p{2cm}|p{2cm}|p{2cm}|p{2cm}|p{2cm}|p{2cm}|p{2cm}| }
    \hline
    \multicolumn{13}{|c|}{ Analysis and Error } \\
    \hline 
    \centering Trial \# & $v_{1i}$ (m/s) & $v_{1f}$ (m/s) & $v_{2i}$ (m/s) & $v_{2f}$ (m/s) & $P_{i}$ system (kgm/s)& $P_{f}$ system (kgm/s) & \% diff in $P$ & $K_{i}$ system (J) & $K_{f}$ system (J) & \% diff in $K$ \\ 
    \hline
    1 & c2 & c3 & c4 & c5 & c6 & c7 & c8 & c9 & c10 & c11 & c12 & c13 \\
    2 & c2 & c3 & c4 & c5 & c6 & c7 & c8 & c9 & c10 & c11 & c12 & c13 \\
    3 & c2 & c3 & c4 & c5 & c6 & c7 & c8 & c9 & c10 & c11 & c12 & c13 \\
    4 & c2 & c3 & c4 & c5 & c6 & c7 & c8 & c9 & c10 & c11 & c12 & c13 \\
    5 & c2 & c3 & c4 & c5 & c6 & c7 & c8 & c9 & c10 & c11 & c12 & c13 \\
    \hline
  \end{tabular}
\end{center}

\begin{center}
  \begin{tabular} { |p{2cm}|p{2cm}|p{2cm}|p{2cm}|p{2cm}| }
    \hline
    \multicolumn{5}{|c|}{ Error } \\
    \hline 
    \centering Trial\\ \# & Theoretical using C as 1 & Theoretical using an $R_1$ \& $R_2$ &  Experimental & \%error \\ 
    \hline
    \# 1& $2.1641 \; sec$ & $1.9957 \; sec$ & $2.1946 \; sec$ & $1.41\%$ \\
    \hline
    \# 2& $1.5392 \; sec$ & $1.4138 \; sec$ & $1.4367 \; sec$ & $6.66\%$ \\
    \hline
    \# 3& $1.5392 \; sec$ & $1.4138 \; sec$ & $1.4367 \; sec$ & $6.66\%$ \\
    \hline
    \# 4& $1.5392 \; sec$ & $1.4138 \; sec$ & $1.4367 \; sec$ & $6.66\%$ \\
    \hline
    \# 5& $1.5392 \; sec$ & $1.4138 \; sec$ & $1.4367 \; sec$ & $6.66\%$ \\
    \hline
  \end{tabular}
\end{center}

\begin{center}
  \begin{tabular} { |p{2cm}|p{2cm}|p{2cm}|p{2cm}|p{2cm}| }
    \hline
    \multicolumn{5}{|c|}{ Results } \\
    \hline 
    \centering Trial # & Theoretical using C as 1 & Theoretical using an $R_1$ \& $R_2$ &  Experimental & \%error \\ 
    \hline
    \# 1& $2.1641 \; sec$ & $1.9957 \; sec$ & $2.1946 \; sec$ & $1.41\%$ \\
    \hline
    \# 2& $1.5392 \; sec$ & $1.4138 \; sec$ & $1.4367 \; sec$ & $6.66\%$ \\
    \hline
    \# 3& $1.5392 \; sec$ & $1.4138 \; sec$ & $1.4367 \; sec$ & $6.66\%$ \\
    \hline
    \# 4& $1.5392 \; sec$ & $1.4138 \; sec$ & $1.4367 \; sec$ & $6.66\%$ \\
    \hline
    \# 5& $1.5392 \; sec$ & $1.4138 \; sec$ & $1.4367 \; sec$ & $6.66\%$ \\
    \hline
  \end{tabular}
\end{center}
% \restoregeometry

\section*{RSS Error}

\begin{center}
  \noindent Greatest contributors are used to calculate RSS error
  \begin{equation*}
      \text{RSS} = \sqrt{ \left( \frac{0.00005kg}{0.2155kg} \right)^2 
      + \left( \frac{0.005 m/s}{0.0m/s} \right)^2}
      \times 100\% = 0.02\%
  \end{equation*}
\end{center}

\end{document}
